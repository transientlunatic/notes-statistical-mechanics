\section{The First Law}
\label{sec:first-law}

\newcommand{\ddr}[1]{\dd{#1}}

We can reach a microscopic understanding of the First Law starting
with the Canonical Distribution,
\begin{equation}
  \label{eq:14}
  P_i = \frac{e^{-\beta E_i}}{Z}
\end{equation}
with the expression for $U = \ev{E} = \sum_i E_i P_i$. For a
reversible change of $E_i$ in a system due to a change in volume,
\begin{align*}
  \dd{u} &= \sum_i E_i \dd{P_i} + \sum_i P_i \dd{E_i} \\
&= \sum_i E_i \dd{P_i} + \sum_i P_i \qty( \pdv{E_i}{V} ) \dd{V} \\
&= \sum_i E_i \dd{P_i} - p \dd{V}
\end{align*}
This is equivalent to 
\begin{equation*}
  \dd{u} &= \ddr{Q} + \ddr{w}
\end{equation*}
for $\ddr{u} = \sum_i E_i \dd{P_i}$, and $\ddr{u}~{rev} = T \dd{S} =
\sum_iE_i \dd{P_i}$, for $S$ the entropy of a system.

\section{Relating $\beta$ and $T$}
\label{sec:relating-beta-t}

We can show that $\beta$ is a function of $T$ by onstruting a canonial
ensemble in which each system is a composite of two subsystems. These
must be in thermal equilibrium, and so the subsystems, $A$ and $B$,
must have the same temperature, $T_A = T_B$. We have the constraints 
\[ \sum_i a_{A,i} = A \qquad \sum_i a_{B,i} = A \]
Then
\[\sum a_{A,i} E_{A,i} + \sum_i a_{B,i} E_{B,i} = E~{tot} \]
The total energy in an ensemble is fixed but can be transferred
between subsystems. These three constraints imply three Lagrange
multipliers on the system, and we end up with
\[ P_{A,i} = \frac{e^{-\beta E_{A,i}}}{Z_A}, \qquad P_{B,i} =
\frac{e^{-\beta E_{B,i}}}{Z_B} \] where, as a result of the total
energy constraint $\beta$ is the same for each expression. If the two
subsystems always have the same $\beta$ and $T$ then 
\[ \beta = \beta(T) \]
Now, consider the function
\begin{equation}
  \label{eq:15}
  H = - \sum_i P_i \log(P_i)
\end{equation}
then, for a change,
\begin{align*}
  \dd{H} &= - \sum_i \dd{P_i} - \sum_i \dd{P_i} \log{P_i} \\
  &= - \sum_i \dd{P_i} -\sum_i \dd{P_i} \qty( log(e^{-\beta E_i}) - \log(z) ) \\
  &= - \sum_i \qty( \dd{P_i} \qty( - \beta E_i - \log(z) + \dd{P_i} )) \\
  &= - \sum_i \dd{P_i} \qty( 1 - \log(z) ) + \beta \sum \dd{P_i} E_i \\
  \text{since } & \sum P_i = 1 \implies \sum \dd{P_i} =0 \\
  &= \beta \sum_i \dd{P_i} E_i \\
  \shortintertext{then, assuming that $\beta T = \text{const} = k~B^{-1}$,}
  k~B \dd{H} &= \dd{s} \\
\therefore s &= k~B H = -k~B \sum_i P_i \log(P_i)
\end{align*}
This is an important but controversial result, but it is a consistent
form for $s$.

Consider a microcanonical system with a total of $\Omega$ states,
which are degenerate. Now assuming that the entropy is
\begin{equation}
  \label{eq:16}
  s = \phi(\Omega)
\end{equation}
and the probability of each microstate is proportional to $1/\Omega$,
taking two systems, $A$ and $B$ then the individual entropies are
\[ s~A = \phi(\Omega~A), \qquad s~B = \phi(\Omega~B) \] Entropy is
extensive, depending upon the total quantity of material in the
system, and so is additive for weakly interacting systems, so
\[ s~{AB} = s~A + s~B \] But the number of states must be
multiplicative, with $\Omega~{AB} = \Omega~A + \Omega~B$, so
\[ S~{AB} = \phi(\Omega~{AB}) = \phi(\Omega~A \Omega~B) =
\phi(\Omega~A) + \phi(\Omega~B) \] From this we can infer that the
relationship must be logarithmic, with
\begin{fequation}[Boltzmann entropy equation]
  \label{eq:17}
  s = k~B \log(\Omega)
\end{fequation}
Now we have $s$ as a function of $k~B H$:
\begin{align*}
  s &= -k~B \sum_i P_i \log(P_i) \\
&= - k~B \sum_i P_i \qty[ - \beta E_i - \log(Z) ]\\
&= k~B \beta \sum_i P_i E_i + k~B \log(z) \sum_i P_i
\intertext{since $\sum_i P_i =1$,}
&= \frac{1}{T} U + k~B \log(Z)
\end{align*}
Thus we have a new thermodynamic quantity,
\begin{fequation}[The Helmholtz free energy]
  \label{eq:18}
  F = U - Ts = -k~B T \log(Z)
\end{fequation}
which represents the amount of work which an isolated temperature at a
temperature $T$ can perform. This is the fundamental connection
between thermodynamics and statistical mechanics.

All of the other useful thermodynamic quantities can be derived using
this relationship, and its differential form,
\begin{equation}
  \label{eq:19}
  \dd{F} = - s \dd{T} - p \dd{V}.
\end{equation}

Thus
\begin{align}
  s = - \eval{\pdv{F}{T}}_V &= \qty( \pdv{T})_V \qty( k~B T \log(Z) ) \\
p = - \eval{\pdv{F}{V}}_T &= - k~B T \qty(\pdv{V})_T \qty( \log(Z) )
\end{align}
Recalling the Maxwell relations for $F=F(T,V,N)$, since $N$ can vary
in the grand canonical regime, we have 
\begin{equation}
  \label{eq:20}
  \dd{F} = - s \dd{T} - p \dd{V} + \mu \dd{N}
\end{equation}
for 
\begin{equation}
  \label{eq:21}
  \mu = \qty(\pdv{F}{N})_{T,V} = - k~B T \eval{ \pdv{V}}_{V,T} \log(Z)
\end{equation}
being the chemical potential.

Now consider a weakly interacting system of particles with a common
temperauture, $T$, then it is possible to define energy states of the
combined system in the form
\[ E~{Ai} + A~{Bj} \] which have a combined partition function,
\[ Z~{AB} = \sum_i \sum_j e^{- \beta E~{Ai} - \beta E~{Bj}} \] Now
consider $Z$ energies in each state, $A$ and $B$, for example,
\begin{align*}
  Z~{AB} &= \sum_{i=1}^2 \sum_{j=1}^2 e^{- \beta E~{Ai} - \beta E~{Bj}} \\
&= e^{-\beta A~1} e^{- \beta B_1} + e^{-\beta A_2} e^{-\beta B_1} \\ & \quad {}+ e^{-\beta A_1} e^{- \beta B_2} + e^{-\beta A_1}e^{-\beta B_2} \\
& = \qty( e^{-\beta E~A_1} + e^{-\beta E~{A_2}}) \qty( e^{-\beta E~{B_2}} + e^{-\beta E~{B_{2}}} ) \\ &= Z~A Z~B
\end{align*}
Thus the partition function of the combined weakly interacting system
is the product of the individual partition functions, and since
\begin{equation}
  \label{eq:22}
  F = - k~B T \log(Z) = -k~B T \log(Z~A Z~B) = F~A + F~B
\end{equation}
This multiplicative property of the partition functions is useful for
a range of complex systems, including the characterisation of
molecules' energies.

\section{Constructing a partition function}
\label{sec:constr-part-funct}

\subsection{A 2-energy-level system with $E_0=0$ and $E_1=E$	}
\label{sec:2-energy-level}

We have 
\[ Z = 1 + \exp(- \beta E) \] so the probability of each state being
occupied is
\begin{align*} P_0 &= \qty( 1+ \exp(-\beta E))^{-1}, \\ P_1 &= \exp(- \beta E) \qty( 1+ \exp(-\beta E))^{-1} \end{align*}
Thus
\[ F = -k~B T \log(1+ \exp(-\beta T)) \]

\subsection{Spin-$\half$ particle in a magnetic field}
\label{sec:spin-half-particle}

The energies of a spin-half particle in a magnetic field, $\vec{B}$,
will be \[ \pm E = \pm \gamma B \] corresponding to spin quantum
numbers $m \pm 1$, for $\gamma$ the projected magnetic moment. Thus
\begin{align*}
  Z &= \exp(\beta E) + \exp(- \beta E) \\ &= 2 \cosh(\beta E) \\
U &= F + Ts \\ s &= \pdv{T} (k~B T \log(Z) ) \\
U &= -k~B T \log(Z) - T \pdv{T} \qty( k~B T \log(Z)) \\ &= - k~B T \log(Z) + k~B T \log(Z) + k~B T^2 \pdv{T} \log(Z) \\ 
\pdv{T} \log(Z) &= \frac{1}{Z} \pdv{Z}{T}= \frac{1}{Z} \pdv{T} \qty(2 \cosh(\frac{E}{k~B T}) ) \\
&= \frac{2}{Z} \sinh( \frac{E}{k~B T}) \qty( - \frac{E}{k~B T^2} ) \\
&= - \frac{E}{k~B T^2} \tanh( \beta E) \\
U &= - E \tanh(\beta E) \\ &= - \gamma \beta \tanh(\frac{ \gamma \beta }{k~B T} )
\end{align*}
The mean magnetic moment can be found as
\begin{align*}
  \ev{m} &= \sum_i^2 P_i m_i = \gamma P_+ + (- \gamma) P_- \\
&= \gamma \qty[ \frac{1}{Z} \exp(\frac{\gamma B}{k~B T}) - \frac{1}{Z} \exp- \frac{\gamma B}{k~B T}] \\
&= 2 \frac{\gamma}{Z} \sinh( \frac{\gamma B}{k~B T} ) \\
&= \gamma \tanh( \frac{\gamma B}{k~B T})
\end{align*}
Thus
\begin{equation}
  \label{eq:1}
  U = - \ev{m} B
\end{equation}
which is what we expect. In the case that $\frac{\gamma B}{k~B T}$ is
very small, so in the case $T \gg 1$ or $B \ll 1$, then
\begin{fequation}[Curie Law]
  \label{eq:2}
  \ev{m} \approx \frac{\gamma^2 B}{k~B T} \sim \frac{1}{T}
\end{fequation}

\section{The harmonic oscillator}
\label{sec:harmonic-oscillator}

From both quantum mechanics and a simple analogy to standing waves in
a fixed volume, the energy of a fixed oscillator is 
\[ E_n = \qty(n+\half) h \nu, \quad \text{for}\ n \in \mathbb{N} \]
with $\nu$ the frequency of the oscillation. Then the partition
function can be found
\begin{align*}
  Z &= \sum^{\infty}_{n=0} \exp( - \frac{(n+\half) h \nu}{k~B T} )\\
&= \exp( - \half \frac{h \nu}{k~B T} ) \sum \exp(- \frac{h \nu}{k~B T})^n 
\intertext{Using the relation $\sum_{n=0}^{\infty} x^a = (1-x)^{-1}$}
&= \frac{\exp(- \half \frac{h \nu}{k~B T})}{1 - \exp( - \frac{h \nu}{k~B T}) }\\
&= \frac{\exp( - \frac{\Theta~\nu}{ 2 T})}{ 1- \exp(-\frac{\Theta~\nu}{T})}
\intertext{having defined the vibrational temperature,
\begin{equation}
  \label{eq:3}
  \Theta = \frac{h \nu}{k~B}
\end{equation}
}
Z &= \frac{1}{\exp( \frac{\Theta~\nu}{2 T}) - \exp( - \frac{\Theta_{\nu}}{2 T}) } \\
&= \frac{1}{2 \sinh( \frac{\Theta~\nu}{2 T} )}
\end{align*}
Armed with the partition function it is possible to find any
thermodynamic property of the system, for example
\[ U = k~B T^2 \dv{t} \log(Z) \]
So
\begin{align*}
  Z &= \exp(- \frac{\Theta~\nu}{2 T}) \qty( 1 - \exp(- \frac{\Theta~\nu}{T}) )^{-1} \\
\log(Z) &= - \frac{\Theta~\nu}{2 T} - \log( 1 - \exp(- \frac{\Theta~\nu}{T} ) ) \\
\dv{T} \log(Z) &= \frac{\Theta~\nu}{2T^2} - \frac{1}{1-\exp(\frac{\Theta~\nu}{T})} \qty[ - \qty( \frac{\Theta~\nu}{T}  ) \exp(- \frac{\Theta~\nu}{T}) ] \\
&= \frac{\Theta~\nu}{2 T^2} + \frac{\exp(- \frac{\Theta~\nu}{T})}{1 - \exp( \frac{\Theta~\nu}{T} )} \frac{\Theta~\nu}{T^2} \\
U &= k~B T^2 \dv{T} \log(Z) \\
&= \half k~B \Theta_{\nu} + k~B \Theta_\nu \frac{1}{\exp( \frac{\Theta~\nu}{T}) - 1}\\
&= \half h \nu + \frac{h \nu}{\exp( \frac{\Theta~\nu}{T}) -1}
\intertext{in the high temperature limit $\Theta~\nu/T \ll 1$}
U & \approx  \half h \nu + \frac{h \nu}{1 + \frac{h \nu}{k~B T} -1} \\
& \approx \half h \nu + k~B T \\
& \approx k~B T
\intertext{which is the classical two-dimensional result. \\In the low temperature limit we assume $\Theta \nu / T \gg 1$}
U & \approx \hal h \nu + h \nu \exp( - \frac{\Theta~\nu}{T}) 
\end{align*}
So as $\theta_{\nu}/T \to \infty$ we get
\begin{equation}
  \label{eq:4}
  U \approx \text{zero point energy}
\end{equation}
In classical thermodynamics the heat capacity, $C_{V}$, is constant,
that is $\dv{t} C_V = 0$ but here
\begin{align*}
  C_V &= \dv{U}{T} \approx h \nu \qty( \frac{\Theta_{\nu}}{T^2}) \exp(- \frac{\Theta_{\nu}}{T}) \\
&= \frac{h \nu \Theta_{\nu}}{T^2} \exp(- \frac{h \nu}{k~B T}) \to 0 \ \text{as} \ T \to 0
\end{align*}


%%% Local Variables: 
%%% mode: latex
%%% TeX-master: "../project"
%%% End: 
